
\documentclass[journal]{IEEEtran}
%
% If IEEEtran.cls has not been installed into the LaTeX system files,
% manually specify the path to it like:
%\documentclass[journal]{./ieeetrans_template/IEEEtran}


% *** GRAPHICS RELATED PACKAGES ***
%
\ifCLASSINFOpdf
  \usepackage[pdftex]{graphicx}
  % declare the path(s) where your graphic files are
  % \graphicspath{{../pdf/}{../jpeg/}}
  % and their extensions so you won't have to specify these with
  % every instance of \includegraphics
  % \DeclareGraphicsExtensions{.pdf,.jpeg,.png}
\else
  % or other class option (dvipsone, dvipdf, if not using dvips). graphicx
  % will default to the driver specified in the system graphics.cfg if no
  % driver is specified.
  \usepackage[dvips]{graphicx}
  % declare the path(s) where your graphic files are
  % \graphicspath{{../eps/}}
  % and their extensions so you won't have to specify these with
  % every instance of \includegraphics
  % \DeclareGraphicsExtensions{.eps}
\fi
% graphicx was written by David Carlisle and Sebastian Rahtz. It is
% required if you want graphics, photos, etc. graphicx.sty is already
% installed on most LaTeX systems. The latest version and documentation can
% be obtained at: 
% http://www.ctan.org/tex-archive/macros/latex/required/graphics/
% Another good source of documentation is "Using Imported Graphics in
% LaTeX2e" by Keith Reckdahl which can be found as epslatex.ps or
% epslatex.pdf at: http://www.ctan.org/tex-archive/info/
%
% latex, and pdflatex in dvi mode, support graphics in encapsulated
% postscript (.eps) format. pdflatex in pdf mode supports graphics
% in .pdf, .jpeg, .png and .mps (metapost) formats. Users should ensure
% that all non-photo figures use a vector format (.eps, .pdf, .mps) and
% not a bitmapped formats (.jpeg, .png). IEEE frowns on bitmapped formats
% which can result in "jaggedy"/blurry rendering of lines and letters as
% well as large increases in file sizes.
%
% You can find documentation about the pdfTeX application at:
% http://www.tug.org/applications/pdftex


% *** Do not adjust lengths that control margins, column widths, etc. ***
% *** Do not use packages that alter fonts (such as pslatex).         ***
% There should be no need to do such things with IEEEtran.cls V1.6 and later.
% (Unless specifically asked to do so by the journal or conference you plan
% to submit to, of course. )

% correct bad hyphenation here
\hyphenation{op-tical net-works semi-conduc-tor}


\begin{document}

% paper title
% can use linebreaks \\ within to get better formatting as desired
\title{Adaptive MIMO Decoder \\ {\Large EE290C Project Final Report}}

% use \thanks{} to gain access to the first footnote area
% a separate \thanks must be used for each paragraph as LaTeX2e's \thanks
% was not built to handle multiple paragraphs
%

\author{Antonio Puglielli and Simon Scott}


% make the title area
\maketitle

%%%% OBJECTIVES SECTION %%%%
\section{Objectives}

\IEEEPARstart{T}{he} objectives of this project are to...

\begin{figure}[!h]
\centering
\includegraphics*[width=5cm, viewport = 300 0 560 130]{images/agilent_mimo.jpg}
\caption{A 2x2 MIMO configuration}
\label{mimo_cartoon}
\end{figure}


%%%% MOTIVATION AND ALGORITHM SECTION %%%%
\section{Motivation for the LMS Decoding Algorithm}

ANTONIO:
Feel free to write whatever you want here. Essentially, this section will probably be the same as the second slide in our presentation.
Probably also copy some of the text from the motivation section of the project proposal.
Also, feel free to change the title.


%%%% COMPARISON OF OTHER DECODERS SECTION %%%%
\section{Comparison to other MIMO Decoding Algorithms}

ANTONIO:
Compare the MATLAB simulations. Figure \ref{ser_snr_different_schemes_static} is for a static purely random channel. Figure \ref{ser_snr_different_schemes_dynamic} is based on IEEE model B and has 3Hz doppler.

\begin{figure}[!h]
\centering
\includegraphics[width=8cm]{images/static_channel_decoder_comparison.png}
\caption{SER vs SNR for different MIMO detection schemes with a static channel}
\label{ser_snr_different_schemes_static}
\end{figure}

\begin{figure}[!h]
\centering
\includegraphics[width=9cm]{images/time_varying_channel_doppler_3.png}
\caption{SER vs SNR for different MIMO detection schemes with a time varying channel (doppler shift of 3Hz)}
\label{ser_snr_different_schemes_dynamic}
\end{figure}


%%%% ARCHITECTURE SECTION %%%%
\section{Hardware Architecture}

Overview description of hardware.

\begin{figure}[!h]
\centering
\includegraphics*[width=9cm, viewport = 0 90 789 600]{images/top_level_arch.pdf}
\caption{The hardware architecture of the LMS Adaptive MIMO Decoder}
\label{top_level_arch}
\end{figure}

\subsection{Matrix Engine}

\begin{figure}[!h]
\centering
\includegraphics*[width=4cm, viewport = 30 640 260 840]{images/matrix_engine.pdf}
\caption{Design of the matrix engine}
\label{matrix_engine}
\end{figure}

\subsection{Channel Estimator Module}

ANTONIO

\subsection{Initialize Weights Module}

ANTONIO

\subsection{Adaptive Decoder Module}

\begin{figure}[!h]
\centering
\includegraphics*[width=8cm, viewport = 0 510 560 810]{images/adaptive_decoder.pdf}
\caption{The adaptive decoder module}
\label{adaptive_decoder}
\end{figure}

\subsection{Evaluation of Bitwidths}

Explain why poor performance at low bitwidth (overflow, causing wraparound). Would be better if saturated. Could also use pseudo-floating point.

\begin{figure}[!h]
\centering
\includegraphics*[width=8cm, viewport = 80 260 560 530]{images/snr_ber_curves_chisel_static.pdf}
\caption{SER vs SNR curves for Chisel implementations with different bitwidths (static channel)}
\label{ser_ber_chisel_static}
\end{figure}

\begin{figure}[!h]
\centering
\includegraphics*[width=9cm, viewport = 80 270 560 530]{images/snr_ber_curves_chisel_doppler.pdf}
\caption{SER vs SNR curves for Chisel implementations with different bitwidths (dynamic channel)}
\label{ser_ber_chisel_dynamic}
\end{figure}


%%%% SYNTHESIS CONSTRAINTS SECTION %%%%
\section{Synthesis Options and Constraints}

Throughput requirement
Levels of parallelism
Latency of seeding matrix, based on 802.11ac specs.


%%%% SYNTHESIS RESULTS SECTION %%%%
\section{Synthesis Results}

Clock, area power
Power breakdown
Can achieve better results if voltage scale.

ANTONIO: if you can, please insert some 2x2 vs 4x4 results here. Mention that 4x4 hardware supports 2x2, but more efficient to generate 2x2 hardware.

ANTONIO: do we want to compare to area/power of other MIMO decoders from the literature?

\begin{figure}[!h]
\centering
\includegraphics*[width=8cm, viewport = 60 250 560 540]{images/power_vs_area.pdf}
\caption{Power vs area for different levels of parallelism}
\label{power_vs_area}
\end{figure}

Conclusion: better performance if faster clock and take many cycles in MAtrix Engine.

\begin{figure}[!h]
\centering
\includegraphics*[width=6cm, viewport = 90 100 660 540]{images/power_breakdown_module.pdf}
\caption{Power of LMS MIMO Decoder by module}
\label{power_breakdown_module}
\end{figure}

Power gating will clearly help, as shown in table below.

\begin{table}[!h]
\caption{Breakdown of power by module}
\label{power_breakdown_table}
\centering
\begin{tabular}{l l l r}
\hline
Module & Submodule & & Power (mW) \\
\hline
Adaptive Decoder & & & 15.1 \\
Initialize Weights & Mat4 Inverse & 3.96 & \\
 & Mat2 Inverse & 24.8 & \\
 & Module total & & 33.7 \\
Channel Estimator & & & 2.13 \\
Matrix Engine & & & 25.4 \\
Queues & & & 1.81 \\
\hline
\em{Total} & & & \em{78.14} \\
Total without & & & \\
initialization modules & & & 42.31 \\
\hline
\end{tabular}
\end{table}


%%%% CONCLUSION AND FUTURE WORK SECTION %%%%
\section{Conclusions and Future Work}

ANTONIO: I'll leave this for you to fill out once you've finished your section.

Things to mention: power gating of initialization modules; adjusting mu (LMS step size) dynamically; producing soft decisions and then adjusting weights based on hard decisions from LDPC decoder; voltage scaling as we reduce the clock rate for higher levels of parallelization; pseudo-floating point or saturating arithmetic in inverse engine to allow fewer bits to be used.


% Can use something like this to put references on a page
% by themselves when using endfloat and the captionsoff option.
\ifCLASSOPTIONcaptionsoff
  \newpage
\fi


% references section

\bibliographystyle{IEEEtran}
\bibliography{refs}

\end{document}


